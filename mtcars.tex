\documentclass[10pt,]{article}
\usepackage[T1]{fontenc}
\usepackage{lmodern}
\usepackage{amssymb,amsmath}
\usepackage{ifxetex,ifluatex}
\usepackage{fixltx2e} % provides \textsubscript
% Set line spacing
% use upquote if available, for straight quotes in verbatim environments
\IfFileExists{upquote.sty}{\usepackage{upquote}}{}
\ifnum 0\ifxetex 1\fi\ifluatex 1\fi=0 % if pdftex
  \usepackage[utf8]{inputenc}
\else % if luatex or xelatex
  \ifxetex
    \usepackage{mathspec}
    \usepackage{xltxtra,xunicode}
  \else
    \usepackage{fontspec}
  \fi
  \defaultfontfeatures{Mapping=tex-text,Scale=MatchLowercase}
  \newcommand{\euro}{€}
\fi
% use microtype if available
\IfFileExists{microtype.sty}{\usepackage{microtype}}{}
\usepackage[margin=0.5in]{geometry}
\usepackage{graphicx}
% Redefine \includegraphics so that, unless explicit options are
% given, the image width will not exceed the width of the page.
% Images get their normal width if they fit onto the page, but
% are scaled down if they would overflow the margins.
\makeatletter
\def\ScaleIfNeeded{%
  \ifdim\Gin@nat@width>\linewidth
    \linewidth
  \else
    \Gin@nat@width
  \fi
}
\makeatother
\let\Oldincludegraphics\includegraphics
{%
 \catcode`\@=11\relax%
 \gdef\includegraphics{\@ifnextchar[{\Oldincludegraphics}{\Oldincludegraphics[width=\ScaleIfNeeded]}}%
}%
\ifxetex
  \usepackage[setpagesize=false, % page size defined by xetex
              unicode=false, % unicode breaks when used with xetex
              xetex]{hyperref}
\else
  \usepackage[unicode=true]{hyperref}
\fi
\hypersetup{breaklinks=true,
            bookmarks=true,
            pdfauthor={},
            pdftitle={Analysis of transmission type influence on MPG},
            colorlinks=true,
            citecolor=blue,
            urlcolor=blue,
            linkcolor=magenta,
            pdfborder={0 0 0}}
\urlstyle{same}  % don't use monospace font for urls
\setlength{\parindent}{0pt}
\setlength{\parskip}{6pt plus 2pt minus 1pt}
\setlength{\emergencystretch}{3em}  % prevent overfull lines
\setcounter{secnumdepth}{0}

%%% Change title format to be more compact
\usepackage{titling}
\setlength{\droptitle}{-2em}
  \title{Analysis of transmission type influence on MPG}
  \pretitle{\vspace{\droptitle}\centering\huge}
  \posttitle{\par}
  \author{}
  \preauthor{}\postauthor{}
  \date{}
  \predate{}\postdate{}




\begin{document}

\maketitle


\section{Summary}\label{summary}

In this report, \texttt{mtcars} data set was analysed in order to
explore the relationship between a set of variables and miles per gallon
(MPG). In particular, influence of transmission type (automatic vs
manual) was considered. The analysis was performed by fitting linear
models for MPG. Analysis showed that there is a significant difference
in MPG between cars with automatic and manual transmissions. However,
the best linear model for MPG as outcome did not include transmission
type as regressor. More specifically, having car weight and number of
cylinders fixed, transmission type does not influence MPG. {[}Report was
prepared with \texttt{knitr}. Source code may be found at
\texttt{https://github.com/vbalys/coursera-regression-project}{]}.

\section{Data}\label{data}

The \texttt{mtcars} data set consists of 32 observations of 11
variables. \texttt{mpg} is miles per galon - outcome that we are
modeling, and \texttt{am} is transmission type (\texttt{0} - automatic,
\texttt{1} - manual) - variable of our interest. The remaining 9
variables are possible regressors for a model explaining \texttt{mpg}
values. Fig. 1 in the Appendix summarises dependencies between
variables.

Boxplots in Fig. 2 suggest that there is a difference in MPG between
transmission types. Indeed, t test confirms that difference between
means (24.39 vs 17.15) is significant (\texttt{p = 0.001374}). However,
this does not necessarily mean that difference in MPG is actually
related to transmission type. It is possible that we observe a result of
confounding factors that are correlated both with MPG and transmission
type. To answer the questions of this research, we have to build linear
model for MPG. And only then we will find out if there is a way to
quantify influence of transmission type.

\section{Analysis}\label{analysis}

Let us first start with a simplistic model where MPG is explained only
by transmission type (\texttt{mpg \textasciitilde{} am}). It is
immediately obvious that this model is underspecified. Adjusted
R-squared ($R^2_{adj}$) value is only 0.33, while residuals vs.~fitted
values plot in Fig. 3 does not show the expected normality of residuals.
This comes as no surprise as transmission type is a factor taking only
two values, therefore model can predict only two values (means of MPG
for each transmission type).

Another candidate is a model that uses all variables as predictors
(\texttt{mpg \textasciitilde{} .}). In this case, we get overspecified
model. Even though $R^2_{adj}=0.8$ and residuals vs.~fitted values plot
in Fig. 4 is much better, none of the coefficients are significant.
Clearly, we have included too much predictors that in various ways
correlate with each other and thus cancel each other out.

\begin{verbatim}
##                Estimate  Std. Error    t value   Pr(>|t|)
## (Intercept) 12.30337416 18.71788443  0.6573058 0.51812440
## cyl         -0.11144048  1.04502336 -0.1066392 0.91608738
## disp         0.01333524  0.01785750  0.7467585 0.46348865
## hp          -0.02148212  0.02176858 -0.9868407 0.33495531
## drat         0.78711097  1.63537307  0.4813036 0.63527790
## wt          -3.71530393  1.89441430 -1.9611887 0.06325215
## qsec         0.82104075  0.73084480  1.1234133 0.27394127
## vs           0.31776281  2.10450861  0.1509915 0.88142347
## am           2.52022689  2.05665055  1.2254035 0.23398971
## gear         0.65541302  1.49325996  0.4389142 0.66520643
## carb        -0.19941925  0.82875250 -0.2406258 0.81217871
\end{verbatim}

Now we have to find a model in between these two extreme ones that would
reasonably explain variation of MPG, have significant coefficients and
approximately normal residuals. We are including transmission type
\texttt{am} as predictor by default, because we are looking for its
influence. For the remaining 9 variables there are 512 ($2^9$)
combinations of including/excluding any of them - clearly too much to
check all of them.

Let us first start with a ``reasonable'' model which we will then try to
improve. If we think from a mechanistical point of view, we can come up
with some clear candidates for important predictors. Weight
(\texttt{wt}) is obviously a factor to consider when talking about MPG.
Acceleration (\texttt{qsec}) is itself an outcome of car design choices,
and using it to predict MPG would be not logical. Similarly, horsepower
(\texttt{hp}) is a result of other factors, so we choose to not include
it in the model. Number of cylinders (\texttt{cyl}) and engine
displacement (\texttt{disp}) are highly related (indeed, correlation is
0.9) to each other and probably interchangeable, therefore we choose to
include only one of them - \texttt{cyl} which is factor variable with
three levels. The remaining ones (rear axle ratio \texttt{drat}, engine
type \texttt{vs}, number of forward gears \texttt{gear} and number of
carburetors \texttt{carb}) are rather obscure ones, therefore we leave
them out from the model. So, we start with initial model
\texttt{mpg \textasciitilde{} am + wt + cyl}.

The resulting linear regression model has significant coefficients for
\texttt{wt} and \texttt{cyl}, $R^2_{adj}=0.81$, and residuals vs fitted
values plot in Fig. 4 looks pretty good with no obvious patterns or
assymetry, but with a couple of possible outliers. As coefficient for
\texttt{am} is not significant ($p=0.89$), we run nested model tests
with ANOVA to check which of variables are needed.

\begin{verbatim}
## Analysis of Variance Table
## 
## Model 1: mpg ~ wt
## Model 2: mpg ~ wt + cyl
## Model 3: mpg ~ wt + cyl + am
##   Res.Df    RSS Df Sum of Sq       F Pr(>F)   
## 1     30 278.32                               
## 2     29 191.17  1    87.150 12.7728 0.0013 **
## 3     28 191.05  1     0.125  0.0183 0.8933   
## ---
## Signif. codes:  0 '***' 0.001 '**' 0.01 '*' 0.05 '.' 0.1 ' ' 1
\end{verbatim}

We get that we improve model by adding \texttt{cyl} variable, but adding
\texttt{am} does not yield significant improvement. Therefore, we get
the final model \texttt{mpg \textasciitilde{} wt + cyl}.

\begin{verbatim}
## 
## Call:
## lm(formula = mpg ~ wt + cyl, data = mtcars)
## 
## Residuals:
##     Min      1Q  Median      3Q     Max 
## -4.2893 -1.5512 -0.4684  1.5743  6.1004 
## 
## Coefficients:
##             Estimate Std. Error t value Pr(>|t|)    
## (Intercept)  39.6863     1.7150  23.141  < 2e-16 ***
## wt           -3.1910     0.7569  -4.216 0.000222 ***
## cyl          -1.5078     0.4147  -3.636 0.001064 ** 
## ---
## Signif. codes:  0 '***' 0.001 '**' 0.01 '*' 0.05 '.' 0.1 ' ' 1
## 
## Residual standard error: 2.568 on 29 degrees of freedom
## Multiple R-squared:  0.8302, Adjusted R-squared:  0.8185 
## F-statistic: 70.91 on 2 and 29 DF,  p-value: 6.809e-12
\end{verbatim}

Finally, we run some diagnostics for the final model (see the last
figure in the Appendix). Both from diagnostic plots and calculated
\texttt{PRESS = resid / (1 - hatvalues)} values we see that
\texttt{Toyota Corolla}, \texttt{Fiat 128} and
\texttt{Chrysler Imperial} are outliers. This does not mean that there
is something wrong with the data, simply MPG for these cars are not
correctly explained by our model.

\section{Conclusions}\label{conclusions}

\begin{enumerate}
\def\labelenumi{\arabic{enumi}.}
\item
  There is a significant difference in MPG depending on transmission
  type. Cars with automatic tranmission have lower MPG than cars with
  manual transmission.
\item
  The best linear model for MPG that we managed to come up did not
  include transmission type as a predictor. This model did include car
  weight and number of cylinders as regressors. It is highly possible
  that transmission exerts its influence via car weight.
\item
  Without linear model with transmission type as a regressor there is no
  way to quantify its influence on MPG.
\end{enumerate}

\section{Appendix}\label{appendix}

\includegraphics{mtcars_files/figure-latex/unnamed-chunk-11-1.pdf}

\includegraphics{mtcars_files/figure-latex/unnamed-chunk-12-1.pdf}

\includegraphics{mtcars_files/figure-latex/unnamed-chunk-13-1.pdf}

\includegraphics{mtcars_files/figure-latex/unnamed-chunk-14-1.pdf}

\includegraphics{mtcars_files/figure-latex/unnamed-chunk-15-1.pdf}
\includegraphics{mtcars_files/figure-latex/unnamed-chunk-15-2.pdf}
\includegraphics{mtcars_files/figure-latex/unnamed-chunk-15-3.pdf}
\includegraphics{mtcars_files/figure-latex/unnamed-chunk-15-4.pdf}

\end{document}
